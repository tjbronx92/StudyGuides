\documentclass[a4paper, 12pt]{article}
\usepackage[english]{babel}
\usepackage{blindtext}
\usepackage{fancyhdr}

\begin{document}
\title{Kubernetes 101 Notes}
\author{\scshape TJ Robinson}
\maketitle

\section{\texttt{kubectl} Commands}
\begin{description}
    \item[kubectl] is the command-line tool used to interact with Kubernetes clusters.
\end{description}

\subsection{\texttt{kubectl} Resources}
\begin{itemize}
    \item \texttt{kubectl api-resources} - List all available resource types in the cluster.
    \item \texttt{kubectl get pods} - List all pods in the current namespace.
    \item \texttt{kubectl get services} - List all services in the current namespace.
    \item \texttt{kubectl get deployments} - List all deployments in the current namespace.
\end{itemize}

\subsection{\texttt{kubectl} Documentation}
\begin{itemize}
    \item \texttt{kubectl explain <resource>} - Show documentation for a specific resource type.
    \item \texttt{kubectl explain <resource>.<field>} - Show documentation for a specific field of a resource.
    \item \texttt{kubectl explain --recursive <resource>} - Show documentation for a resource and all its fields recursively.
\end{itemize}

\subsection{\texttt{kubectl apply}}
\begin{itemize}
    \item \texttt{kubectl apply -f <file.yaml>} - Apply a configuration file to create or update resources in the cluster.
    \item \texttt{kubectl apply -k <directory>} - Apply a directory of configuration files using Kustomize.
\end{itemize}

\end{document}