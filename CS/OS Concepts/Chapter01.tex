\documentclass[a4paper,12pt,twoside]{article}
\usepackage[a4paper]{geometry}
\usepackage[english]{babel}
\usepackage{fancyhdr}
\usepackage{graphicx}

\renewcommand{\footnoterule}{\noindent\smash{\rule[3pt]{\textwidth}{0.4pt}}}
\fancyhf{}
\fancyhead[LE]{\scshape\nouppercase{\leftmark}}
\fancyhead[RO]{\nouppercase{\rightmark}}
\fancyfoot[LE,RO]{\thepage}
\pagestyle{fancy}

\begin{document}

\title{
    \includegraphics[width=10cm, trim={5cm 4cm 5cm 5cm}]{img/embryo.png}\\
    Operating System Concepts
    \\ 
    Chapter 01 Notes
    }
\author{T.J. Robinson}
\maketitle

\section{What is an Operating System}
\begin{description}
    \item [Operating System:] software that manages the hardware of a computer system. It sits between the system's user and hardware.
\end{description}

\subsection{What does the OS do?}
\begin{itemize}
    \item The OS provides an environment for programs to run.
    \item The OS allocates system resources (ex. CPU, RAM) to programs
    \item The OS manages the execution of programs to prevent errors and improper use of the system.
\end{itemize}
\indent{An computer system is devided into 4 components:}
\begin{enumerate}
    \item The \textbf{Hardware} is basic computing resources like CPU(s), RAM, and I/O Devices.
    \item \textbf{Application(s)} are programs \textsl{The User} interacts with like web browsers, softphones, or media players.
    \item The \textbf{Operating System} orchestrates how applications interact with hardware requests by applications
    \item \textbf{Users} access computer systems to utilize applications.
\end{enumerate}

\subsection{Computer System Organization}

\subsubsection{Interupts}


\end{document}