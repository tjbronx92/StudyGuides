\documentclass[a4paper,twoside]{article}
\usepackage[a4paper,inner=1.5cm,outer=3cm,top=.75cm,bindingoffset=0.75cm]{geometry}
\usepackage[english]{babel}
\usepackage{fancyhdr}
\usepackage{graphicx}

\renewcommand{\footnoterule}{\noindent\smash{\rule[3pt]{\textwidth}{0.4pt}}}
\fancyhf{}
\fancyhead[LE]{\scshape\nouppercase{\leftmark}}
\fancyhead[RO]{\nouppercase{\rightmark}}
\fancyfoot[LE,RO]{\thepage}
\pagestyle{fancy}

\begin{document}
\title{
    \includegraphics[width=4.5cm,trim=8cm 7cm 4cm 4cm]{img/AlmaLinux-Logo.png}
    \\
    Linux Administration Notes
}
\author{\scshape T.J. Robinson}
\maketitle

\section{File Management}
\subsection{File Permissions}
{\begin{description}
\item [File Permissions] are used to prevent unauthorized access by \textsl{users} to files and directories
\end{description}}

\subsection{Permission Classes}
\begin{description}
    \item[Permission Classes] unique cataegorizes utilized by the kernel to maintain file security via access rights. 
\end{description}
Users are assigned to 3 catagories:
\begin{enumerate}
    \item \textbf{User (u)}
    \item \textbf{Group (g)}
    \item \textbf{Other (o)}
    \item \textbf{All (a)} - represents all 3 classes
\end{enumerate}

\subsection{Permission Types}
There are 3 types of permission bits:
\begin{enumerate}
    \item \textbf{Read (r)} - view and copy
    \item \textbf{Write (w)} - modify
    \item \textbf{Execute (x) }- run
    \item \textbf{null (-)} - permission not granted
\end{enumerate}

\subsection{Permission Modes}
\begin{enumerate}
    \item Append permission bit \textbf{(+)} 
    \item Revoke permission bit \textbf{(-)} 
    \item Assign permission bit \textbf{(=)} 
\end{enumerate}



\end{document}