\documentclass[a4paper,twoside]{article}
\usepackage[a4paper,inner=1.5cm,outer=3cm,top=2cm,bindingoffset=0.75cm]{geometry}
\usepackage[english]{babel}
\usepackage{fancyhdr}
\usepackage{graphicx}

\renewcommand{\footnoterule}{\noindent\smash{\rule[3pt]{\textwidth}{0.4pt}}}
\fancyhf{}
\fancyhead[LE]{\scshape\nouppercase{\leftmark}}
\fancyhead[RO]{\nouppercase{\rightmark}}
\fancyfoot[LE,RO]{\thepage}
\pagestyle{fancy}

\begin{document}
\title{
    \includegraphics[width=4.5cm,trim=8cm 7cm 4cm 4cm]{img/AlmaLinux-Logo.png}
    \\
    Linux Administration Notes
}
\author{\scshape T.J. Robinson}
\maketitle

\section{File Management}
\subsection{File Permissions}
{\begin{description}
\item [File Permissions] are used to prevent unauthorized access by \textsl{users} to files and directories
\end{description}}

\subsection{Permission Classes}
\begin{description}
    \item[Permission Classes] unique cataegorizes utilized by the kernel to maintain file security via access rights. 
\end{description}
Users are assigned to 3 catagories:
\begin{enumerate}
    \item \textbf{User Owner (u)}
    \item \textbf{Group (g)}
    \item \textbf{Other (o)}
    \item \textbf{All (a)} - represents all 3 classes
\end{enumerate}

\subsection{Permission Types}
There are 3 types of permission bits:
\begin{enumerate}
    \item \textbf{Read (r)} - view and copy
    \item \textbf{Write (w)} - modify
    \item \textbf{Execute (x) }- run
    \item \textbf{null (-)} - permission not granted
\end{enumerate}

\subsection{Permission Modes}
\begin{enumerate}
    \item Append permission bit \textbf{(+)} 
    \item Revoke permission bit \textbf{(-)} 
    \item Assign permission bit \textbf{(=)} 
\end{enumerate}

\subsection{Modifying Permissions}
\begin{description}
    \item [chmod] is used to change permissions of files \& diectories
\end{description}

\subsubsection{Symbolic vs. Octal Notation}
\begin{itemize}
    \item \textbf{Symbolic Notation} uses letters (ex. \textbf{u,g,o}) \& sumbols (ex. \textbf{+,-,=}) to modify permissions.
    \item \textbf{Octal Notation} uses 3-digit numbering (ex. \textbf{766}) to modify permissions.
\end{itemize}

\subsection{Default Permission}
\begin{description}
    \item [umask] is used to set default permissions on a file without modify permissions on existing files and directories.
\end{description}

\begin{itemize}
\item The default \textsl{umask} value for all users including the root user is \textbf{0022}.
\item The default initial permission value for files is \textbf{666} \& \textbf{777} for directories.
\end{itemize}

\subsection{Calulating Default Permission}
Calculating default permissions for files: \\
\noindent{Initial Permission \indent\indent{ 666} }\\
umask \indent\indent\indent\indent\indent{- 022} \\
================== \\
Default Permission \indent\indent{044}\\

\noindent{Calculating default permissions for directories:}\\
\noindent{Initial Permission \indent\indent{ 777}} \\
umask \indent\indent\indent\indent\indent{- 022} \\
================== \\
Default Permission \indent\indent{055}

\subsection{Special File Permission}
There are 3 Special Permission Bits that can be configured for binary files and directories:
\begin{enumerate}
    \item \textbf{SETUID} (SET User IDentifier) - applied to binary executable files at the \textsl{user owner (u)} level. It gives non-owners the same file permissions as the user owner.
    \item \textbf{SETGID} (SET Group IDentifier) - applied to binary executable files at the \textsl{user owner (u)} level. It gives non-owners and group memebers the same file permissions as the user \& group owner.
    \item \textbf{Sticky Bit} - is set on public directories to prevent other users from deleteing or moving files.
\end{enumerate}

\subsection{File Searching}
\begin{description}
    \item [find] is the command used to search for files on a Linux System and perform actions on found files.
\end{description}
After invoking the find command, the first option is the location path to search (ex. current (.), /tmp, /home/).

\subsection{\texttt{find} Command Options}
\begin{itemize}
    \item use \textbf{-iname} to search for files that \textsl{begins} with a string.
        \begin{description}
            \item [example input:] \texttt{find /dev -iname usb*}
        \end{description}
        \indent\textbf{example output:} \\ 
        \texttt{/dev/usb1} \\
        \texttt{/dev/monusb0} \\
        \texttt{/dev/monusb1}
    \item use \textbf{-size} to search for files by size
        \begin{itemize}
            \item use (-) to find items smaller then designated size
                \begin{description}
                    \item [example input:] \texttt{find /dev -size -2M}
                \end{description}
            \item use (+) to find items larger then designated size
                \begin{description}
                    \item [example input:] \texttt{find /dev -size +2M}
                \end{description}
        \end{itemize}
            \item find files owened by a specific user (\textsl{daemon}) and exclude specific group (\textsl{user1}).
                \begin{description}
                    \item [example input:] \texttt{find /dev -user daemon -not -group user1}
                \end{description}
    \item use \textbf{-type} to seach by filetype (d=directory, f=file)
        \begin{description}
            \item[example input:] \texttt{find /usr -type d -name src}
        \end{description}
    \item use \textbf{-maxdepth} to seach set maximum subdirectory depth to search
        \begin{description}
            \item[example input:] \texttt{find /home -maxdepth 3 -type f -name src}
        \end{description}
\end{itemize}

\begin{figure}
    \centering
    \includegraphics[width=9cm]{img/findsyntax.png}
    \caption{\texttt{find} Command Syntax}
\end{figure}

\subsubsection{Using the \texttt{-exec} and \texttt{-ok} options}
\begin{itemize}
    \item \texttt{-exec} is used to perform actions on the files found by \texttt{find}.
    \item \texttt{-ok} is the same as \texttt{-exec}, but requires user confirmation to execute.
\end{itemize}

\begin{description}
    \item[example input:] \texttt{find /Documents -type f -name BLS* -exec ls -ld \{\} \textbackslash ;}
\end{description}
\begin{itemize}
    \item (\textbf\{\textbf\}) represents each file found
    \item (\textbf{;}) terminates the command.
    \item (\textbf{\textbackslash}) is used to escape (\textbf{;})
\end{itemize}

\section{Linux Processes and Job Scheduling}

\subsection{Processes \& Priorities}
\begin{description}
    \item[process] a unit for provisioning system resources. It is any program, command, or application running on the system.
    \item[daemon] critical system processes that startup automatically and run in the background.
\end{description}
\begin{itemize}
    \item One parent process can spawn one or many child processes and passes attributes to them during creation (i.e., nice score). 
    \item A \textsl{Process Identifacation Number (PID)} is assigned to each process.
    \item The PID is utilized by the kernel to manage the process during it's lifespan.
\end{itemize}

\subsection{Process States}
A process can jump from one operating state to another thoughout it's lifespan. \\
Every process is in one of the \textbf{5 Basic Operating States:}
\begin{enumerate}
    \item \textbf{running} - the process is currently being executed on the system.
    \item \textbf{sleeping} - the process is waiting for input from the user or other source.
    \item \textbf{waiting} - input has been recieved by the process and it is waiting to run.
    \item \textbf{stopped} - the process has been halted and will not run until a signal is recieved to change it's state.
    \item \textbf{zombie} - The process is \textsl{dead}, aka \textsl{defunt}. There is an entry in the process table, but the process takes up no reources (i.e, CPU).
\end{enumerate}

\begin{figure}
    \centering
    \includegraphics[width=9cm]{img/processstates.png}
    \caption{Process States}
\end{figure}

\subsection{Viewing and Monitoring System Processes using \texttt{ps} \& \texttt{top}}
\begin{itemize}
    \item \texttt{ps} (process status)
        \begin{itemize}
            \item Useful Commands \& Options
                \begin{itemize}
                    \item \texttt{ps -as} output all processes and include file size.
                    \item pipe \texttt{ps} output to \texttt{grep} to filter output.
                \end{itemize}
        \end{itemize}
    \item \texttt{top} (table of processes)
\end{itemize}

\subsection{Process Niceness \& Priority}
\begin{itemize}
    \item the \texttt{nice} command can launch a program at a non-default priority.
        \begin{itemize}
            \item nice can also be used to confirm default \texttt{nice} score on RHEL systems.
        \end{itemize}
    \item the \texttt{renice} command is utilized to alter the priority of running processes.
    \item A process's execution priority is determined by the nice score assigned to it when it spawned. There are 40 niceness scores from \textsl{-20 (best)} to \textsl{19 (worst)}.
    \item Higher Nice Score \textbf{=} More CPU Attention
    \item The default \texttt{nice} score for a process is \texttt{0}.
    \item Child process inherit the nice score of it's parent (calling) process.
\end{itemize}

\section{System Date and Time}
\begin{description}
    \item[timedatectl] app used to get \& change system timezone
\end{description}

\subsection{View System Date \& Time}
\begin{itemize}
    \item use \texttt{date} to view system date and time
    \item use \texttt{timedatectl} to view system timezone
\end{itemize}

\subsection{Edit System Date \& Time}
\begin{itemize}
    \item use \texttt{timedatectl set-timezone} to update system timezone
    \item edit \texttt{/etc/systemd/timesyncd.conf} file to update sytem NTP pools.

\end{itemize}

\section{Job Scheduling}
\begin{description}
    \item[job scheduling] allows users to run a command in the future. 
    \item [atd] is the service daemon used to schedule a one time task in the future.
    \item [crond] is the service daemon used to schedule reptitive tasks.
        \begin{itemize}
            \item during system boot the crond daemon reads jobs from \texttt{/var/spool/cron/crontabs} (user cron jobs) \& \texttt{/etc/cron.d} (system cron jobs).
            \item \texttt{crond} loads scheduled jobs into memory at boot and scans the files regularly thereafter.
        \end{itemize}
    \item [crontab] linux app used to edit the \texttt{ /etc/contab} file.
\end{description}

\subsection{Crontab Usage}
\begin{description}
    \item[crontab -e] edit \texttt{/etc/crontab}
    \item[crontab -i] view scheduled cronjobs 
    \item[atq] app to view all scheduled jobs for current user.
\end{description}

\begin{figure}
    \centering
    \includegraphics[width=15cm]{img/cron/ex-crontab-out.png}
    \caption{\texttt{/etc/crontab} File}
\end{figure}

\section{IPtables}



\end{document}