\documentclass[a4paper,12pt]{article}
\usepackage[a4paper, inner=1.5cm, outer=3cm, top=1cm, bottom=3cm, bindingoffset=0.5cm]{geometry}
\usepackage[english]{babel}
\usepackage{fancyhdr}
\usepackage{sectsty}
\usepackage{textcomp}

\sectionfont{\normalfont\scshape}
\subsectionfont{\normalfont\scshape}

\fancyhf{}
\fancyhead[LE]{\scshape\nouppercase{\leftmark}}
\fancyhead[RO]{\nouppercase{\rightmark}}
\fancyfoot[LE,RO]{\thepage}
\pagestyle{fancy}

\begin{document}
\title{Prometheus Cheatsheet}
\author{\scshape TJ Robinson}
\maketitle

\section{Metric vs. Time-Series}
\begin{description}
    \item[Metric] is a numerical value that represents a specific measurement at a specific point in time.
    \item[Time Series] is a sequence of data points, typically ordered by time, that represents the evolution of a metric over time.
        \begin{itemize}
            \item every metric \& label combo is a time series.
        \end{itemize}
\end{description}

\section{Prometheus TSDB}
\begin{description}
    \item[Prometheus TSDB] is a time series database designed for storing and querying time series data, particularly metrics data collected from various sources.
\end{description}

\subsection{Head block}
\begin{description}
    \item[Head block] is the in-memory storage component of Prometheus TSDB that holds the most recent time series data before it is flushed to disk.
        \begin{itemize}
            \item Each chunk for a time series in the head block typically contains data points for a specific time range and can grow to 120 samples before it is flushed to the active chunk's segment file in the \texttt{chunks\_head} directory on the disk.
            \item Each head chunk segment file can grow to 128 MB before it is finalized and a new segment file is created for that chunk.
            \item When the head block reaches 1.5x the minimum block size (default 2 hours), a new block is created, and the current head block is flushed to disk as a finalized block.
        \end{itemize}
\end{description}

\subsection{WAL}
\begin{description}
    \item[WAL (Write-Ahead Log)] is a log file that records all changes made to the head block before they are applied, ensuring data durability and consistency in case of crashes.
    \begin{itemize}
        \item The WAL prevents loss of recent data by allowing Prometheus to replay the log and restore the head block to its last known state upon restart.
        \item WAL segments are stored in the \texttt{wal} directory and are named sequentially (e.g., 00000000, 00000001, etc.).
        \item Each WAL segment can grow up to 512 MB before a new segment is created
    \end{itemize}
\end{description}

\section{promQL (Prometheus Query Language)}
\begin{description}
    \item[promQL]  is the query language used to query metrics stored in Prometheus.
\end{description}

PromQL queries can evaluate to one of four types of results:

\begin{itemize}
    \item Instant vector: A set of time series containing a single sample for each time series, all sharing the same timestamp.
    \item Range vector: A set of time series containing a range of data points over time for each time series.
    \item Scalar: A single numerical value. (i.e., 177.6)
    \item String: A single string value. (i.e., "Primus Sux")
\end{itemize}

\subsection{Selectors \& Matchers}

\subsubsection{Selctors}
    \begin{itemize}
        \item timeseries
        \item \underline{Range Vector Selector} will returns all values for a metric over a time period
    \end{itemize}

\subsubsection{Label Matchers}
    \begin{itemize}
        \item Label matchers allow you to return metrics for a subset of time series.
            \begin{description}
                \item[=] Exact match on a label value.
                \item[!=] Negative equality matcher - return time series that do not have specified label.
                \item[=\texttildelow] Regualar Expression Matcher - utilize \texttt{regex} to match time series.
                \item[!\texttildelow] Negative Regualar Expression Matcher - return time series that do not match \texttt{regex} expression
            \end{description}
        \item multiple seletors can. be used in a query by seperating them with a comma.
    \end{itemize}

\subsection{Modifiers}
\begin{description}
    \item[offset modifiers] allows you top retrieve historc data. 
    \item[@ modifier] allows you to retrieve data from a specific point in time. 
\end{description}

\begin{itemize}
    \item \textbf{offset} and \textbf{@} modifiers can be combined in a query (order does not matter).
    \item \textbf{Range vectors} can also be combined with \textbf{offset} and \textbf{@} modifiers.
\end{itemize}

\subsection{Operators}

\subsubsection{Arithmatic Operators}
    \begin{description}
        \item[Arithmatic Operators] allow users to perform basic math operations
    \end{description}

\subsubsection{Bool Operators}
    \begin{description}
        \item[Bool Operators] 
    \end{description}

\subsubsection{Logical Operators}

\subsection{Vector Matching}
Rules for vector matching:
\begin{enumerate}
    \item \textsl{All} \textsl{labels \underline{must} be identical} between vectors samples to match. 
\end{enumerate}

\subsubsection{Vector Matching Keywords}

\end{document}