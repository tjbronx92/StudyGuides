\documentclass[a4paper,12pt]{article}
\usepackage[english]{babel}
\usepackage{fancyhdr}

\fancyhf{}
\fancyhead[LE]{\scshape\nouppercase{\leftmark}}
\fancyhead[RO]{\nouppercase{\rightmark}}
\fancyfoot[LE,RO]{\thepage}
\pagestyle{fancy}

\begin{document}
\title{Prometheus Cheatsheet}
\author{\scshape TJ Robinson}
\maketitle

\section{\normalfont\textsc{promQL}}
\begin{description}
    \item[promQL]  is the query language used to query metrics stored in Prometheus.
\end{description}

PromQL queries can evaluate to one of four types of results:
\begin{itemize}
    \item Instant vector: A set of time series containing a single sample for each time series, all sharing the same timestamp.
    \item Range vector: A set of time series containing a range of data points over time for each time series.
    \item Scalar: A single numerical value.
    \item String: A single string value.
\end{itemize}



\end{document}