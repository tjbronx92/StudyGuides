\documentclass[a4paper,12pt]{article}
\usepackage[english]{babel}
\usepackage{blindtext}

\begin{document}
\title{\underline{eBPF Notes}}
\author{TJ Robinson}
\maketitle

\section{What is eBPF?}
\begin{itemize}
    \item eBPF (extended Berkeley Packet Filter)
\end{itemize}

\subsection{Logs vs. Metrics vs. Observability}
\begin{itemize}
    \item Logs: Detailed, unstructured, aggregated data about individual events.
    \begin{itemize}
        \item  Useful for troubleshooting and forensic analysis.
        \item Can be voluminous and harder to analyze at scale.
    \end{itemize}
    \item Metrics: Aggregated, structured data for to monitor a program or system performance at a specific point in time.
    \begin{itemize}
        \item Useful for identifying trends and triggering alerts.
        \item Typically less detailed but more efficient to store and query.
    \end{itemize}
    \item Observability: Combines logs, metrics, and traces to provide a comprehensive view of system behavior.
    \begin{itemize}
        \item Capacity to ask arbitrary questions and recieve complex answers about a system's state.
    \end{itemize}
\end{itemize}

\subsection{Namespaces \& Cgroups}
Both are fundamental for containerization and resource management in Linux. 
\begin{itemize}
    \item Namespaces: Isolate system resources for processes (e.g., PID, network, mount).
    \begin{itemize}
        \item inside a namespace, you experience the operating system like there were no other tasks running on the computer
    \end{itemize}
    \item Cgroups: Control and limit resource usage (CPU, memory, I/O) for process groups.
    \begin{itemize}
        \item gives you fine grain control over resource usage like CPU, disk I/O, network, and etc.
    \end{itemize}
\end{itemize}

Tracepoints are static marks in the kernel code that can be used to inject code to inspect the kernel's execution.

\end{document}