\documentclass[a4paper,12pt]{article}
\usepackage[english]{babel}
\usepackage{blindtext}
\usepackage[a4paper, inner=1.5cm, outer=3cm, top=2cm, bottom=3cm, bindingoffset=0.5cm]{geometry}


\begin{document}
\title{\underline{e\scshape{BPF Notes}}}
\author{\scshape{TJ Robinson}}
\maketitle

\section{What is eBPF?}
\begin{itemize}
    \item eBPF (extended Berkeley Packet Filter)
\end{itemize}

\subsection{Logs vs. Metrics vs. Observability}
\begin{itemize}
    \item Logs: Detailed, unstructured, aggregated data about individual events.
    \begin{itemize}
        \item  Useful for troubleshooting and forensic analysis.
        \item Can be voluminous and harder to analyze at scale.
    \end{itemize}
    \item Metrics: Aggregated, structured data for to monitor a program or system performance at a specific point in time.
    \begin{itemize}
        \item Useful for identifying trends and triggering alerts.
        \item Typically less detailed but more efficient to store and query.
    \end{itemize}
    \item Observability: Combines logs, metrics, and traces to provide a comprehensive view of system behavior.
    \begin{itemize}
        \item Capacity to ask arbitrary questions and recieve complex answers about a system's state.
    \end{itemize}
\end{itemize}

\subsection{Namespaces \& Cgroups}
Both are fundamental for containerization and resource management in Linux. 
\begin{itemize}
    \item Namespaces: Isolate system resources for processes (e.g., PID, network, mount).
    \begin{itemize}
        \item inside a namespace, you experience the operating system like there were no other tasks running on the computer
    \end{itemize}
    \item Cgroups: Control and limit resource usage (CPU, memory, I/O) for process groups.
    \begin{itemize}
        \item gives you fine grain control over resource usage like CPU, disk I/O, network, and etc.
    \end{itemize}
\end{itemize}

Tracepoints are static marks in the kernel code that can be used to inject code to inspect the kernel's execution.

\section{BPF Program Types}

\subsection{Socket Filter Programs}
\begin{itemize}
    \item Attach to network sockets to filter packets.
    \item Commonly used for packet filtering and monitoring.
    \item You can not modify the packets, only accept, drop, forward, or observe them.
\end{itemize}

\subsection{Kprobes and Uprobes}
\begin{itemize}
    \item Kprobes: Attach to kernel functions to trace and monitor kernel events.
    \item Uprobes: Attach to user-space functions to trace and monitor application events.
\end{itemize}

\subsection{XDP (eXpress Data Path) Programs}
\begin{itemize}
    \item Attach to the earliest point in the network stack.
    \item Used for high-performance packet processing.
    \item Can modify or drop packets before they reach the kernel networking stack.
\end{itemize}
\end{document}